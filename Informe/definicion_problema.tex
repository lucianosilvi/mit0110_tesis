
\section{Definición del problema}
% Cuál es el problema
% Por qué es importante
% Qué vamos a hacer al respecto
% Por qué lo abordamos de la forma en que lo hacemos
Los sistemas de respuesta a preguntas son un área naciente del procesamiento del lenguaje natural y particularmente del área de recuperación de información.

\citet{gupta_survey} destacan que existen dos formas principales de buscar la respuesta a una pregunta de un usuario. La primera de ellas consiste de encontrar similitudes semánticas o sintácticas entre la pregunta y documentos de texto que pueden contener evidencias para la respuesta. La segunda, que abordaremos durante este trabajo, traduce la pregunta a un lenguaje formal para luego realizar consultas a una base de datos.

La reciente posibilidad de manejo de grandes volúmenes de datos ha permitido la formación de grades bases de conocimiento públicas y disponibles online. Estas web semánticas u ontologías cambian ampliamente el paradigma utilizado hasta el momento, ya que estructuran los datos y permiten entonces extraer relaciones complejas entre sus entidades, como plantea \citet{ou_entailement}.

Sin embargo, el primer paso para la resolución de una pregunta es la formalización de la misma a partir del texto ingresado por el usuario, independientemente del método de extracción de información empleado a continuación. Aún así, la mayoría de los sistemas se centran en la búsqueda de la respuesta más que en la correcta interpretación de la pregunta, y en general se limitan a textos cortos y preguntas puntuales.
%% http://delivery.acm.org/10.1145/1080000/1073092/p41-ravichandran.pdf?ip=201.212.190.199&id=1073092&acc=OPEN&key=4D4702B0C3E38B35.4D4702B0C3E38B35.4D4702B0C3E38B35.6D218144511F3437&CFID=407135832&CFTOKEN=92333839&__acm__=1408457987_7e50476334404266ebcb5c987e916bf4
%%Learning Surface Text Patterns for a Question Answering System 2002
%% Problemas que en general enfrentan los sistemas de respuesta a preguntas
%% -- Probablemente no manejan el concepto de respuestas múltiples a una pregunta, como una colección

Una aproximación simple a este problema es la de Quepy, un framework de traducción automática de preguntas en lenguaje natural a un lenguaje de consultas formalizado. El usuario define una serie de plantillas para cada tipo de pregunta esperado y su correspondiente interpretación en la base de conocimiento elegida. Los lenguajes soportados hasta el momento son MQL y SPARQL; ambos permiten consultas posteriores a FreeBase y DBPedia.

Aunque este sistema está diseñado para simplificar la tarea de construcción de un sistema de respuesta, el trabajo necesario para lograr cobertura es todavía prohibitivo por varios motivos:
\begin{itemize}
    \item Las plantillas deben ser desarrolladas por un experto de forma individual.
    \item El poder expresivo de las preguntas que soporte el sistema es lineal con respecto a la cantidad de plantillas generadas.
    \item Existe redundancia de información. Por ejemplo, para las preguntas ``Who are the presidents of Argentina?'' y ``Who are the children of the presidents of Argentina?'' se necesitan dos plantillas que contienen la misma información para resolver ``presidents of X''.
    \item Existen numerosas preguntas que son equivalentes y que no necesariamente se representan con la misma plantilla. Por ejemplo las preguntas ``Where is Angelina Jolie from?'' y ``Where was Angelina Jolie born?'' tienen escencialmente la misma semántica.
    \item Debido a las grandes variaciones del lenguaje natural, se requiere un anotador experto para lograr una cobertura completa de todas las reformulaciones para una misma semántica.
\end{itemize}

% active learning.
% por qué active learning
% feature selection
% por qué feature selection
% active learning sobre feature selection
% dualist
% por qué tiene sentido
De todas las dificultades anteriores nos enfocaremos en la última de ellas ya que la consideramos prioritara y, al solucionarla, podemos ampliar la cobertura de los sistemas construidos sobre Quepy significativamente.

Cada plantilla de Quepy es una expresión regular que combina distintos tipos de caracteríticas como etiquetas POS y lemas, lo que permite al sistema identificar la semántica de la pregunta únicamente en base a su sintáxis. Ante una concordancia de una pregunta ante una plantilla, Quepy genera una consulta en el lenguaje configurado agregando la información de la entidad a la cual se hace referencia. Por lo tanto, cada plantilla está asociada a una interpretación fija.

% Agregar más datos con grafiquitos sobre lo que hace quepy.

Además de ello, al ser expresiones regulares estos patrones no tienen flexibilidad y dependen fuertemente del analizador sintáctico y POS tagger que utilicen. Nuestra propuesta es aplicar un clasificador automático sobre las preguntas donde cada clase es una interpretación de Quepy. De esta forma, podemos ligar muchas más reformulaciones de la misma pregunta a su correspondiente semántica y lograr mayor versatilidad para el sistema.

% Esto va acá o dentro de la parte de marco teórico?
Este enfoque de encontrar reformulaciones de una misma pregunta está enmarcado dentro del reconocimiento de implicaciones textuales y ha sido utilizado previamente para sistema de respuesta a preguntas del usuario. \citet{ou_entailement} utilizan esta técnica tomando como base preguntas modelo construidas automáticamente desde la ontología, y se centran también en la composición de patrones simples para formar otros más complejos. Sin embargo, se limitan a un dominio muy restringido que permite formar texto en lenguaje natural desde las relaciones formales entre las entidades, lo cual sería dificultoso en ontologías complejas como FreeBase. \citet{rui_relations} explican otros posibles usos de identificar estas relaciones entre las preguntas para sugerencia de preguntas relacionadas o útiles para el usuario.

La originalidad de nuestra aplicación se base en utilizar como características las concordancias parciales con las plantillas de Quepy predefinidas por un usuario. Consideramos que identifican claramente los aspectos relevantes que indican la correcta interpretación de la pregunta, y como tal son mejores representaciones. Una gran parte de los experimentos a realizar consistirá en medir el beneficio de esta nueva representación con respecto a las que comunmente se utilizan para tareas de clasificación de texto. Es lógico pensar entonces en utilizar técnicas desarrolladas para la selección e ingeniería de características.

Para evaluar nuestro sistema consideramos que comenzar con pocos patrones predefinidos nos ayudaría a percibir con más exactitud qué mejora podía generar el clasificador en Quepy. Por ello, y debido a que no existen grandes corpus etiquetados para reformulaciones de preguntas, planteamos que un enfoque de aprendizaje activo es lo más adecuado. El aprendizaje activo, como describe \citet{settles_active_learning_survey}, permite entrenar el aprendedor con menor cantidad de instancias y es beneficioso cuando se cuenta con muchos ejemplos no etiquetadas pero donde la mayoría no son relevantes. En nuestro entorno en particular se da este fenomeno, debido a que en un corpus no anotado estándar pocas de las preguntas caerán dentro de alguna de las clases semánticas de los patrones iniciales.

Un enfoque novedoso que combina todos los conceptos anteriores es el de \citet{dualist} en Dualist. Esta herramienta optimiza el aprendizaje activo no solo preguntado al usuario sobre instancias sino también sobre características de las mismas que las asocian a una clase. Junto con este desarrollo también incluye una serie de investigaciones sobre el rendimiento de tareas de clasificación con usuarios reales y simulados.

% Ampliar un poco más sobre dualist