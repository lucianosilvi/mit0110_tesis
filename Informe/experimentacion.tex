\section{Entorno de experimentación}

\subsection{Ejemplos seleccionados}

% Descripción de los corpus.
	% Etiquetado
		% Las semillas son muy importantes. Deberían ser elegidas al azar?
		% La muestra debería ser representativa de la población? Aproximadamente un 10% de las preguntas son reconocidas por quepy, deberíamos incluir esto en el set de entrenamiento?
	% No etiquetado
	% Testing
		% Las que reconoce quepy?
		% 500 preguntas está bien?
Distribución actual del corpus:
Quepy questions 115
	Recognized 58
	Unrecognized 57
Other questions 6658
	Labeled 607
	Unlabeled 6051

Test corpus 250
Training corpus 165
Unlabeled corpus 6358

\subsection{Experimentos realizados}

% Automáticos sin usuarios
% Con usuarios

\subsubsection{Métricas utilizadas}
\begin{description}
    \item[Precisión] Mediremos cuántas preguntas del corpus de testing fueron etiquetadas correctamente por el clasificador.
    \item[Precisión en preguntas reconocidas] Debido a la gran cantidad de preguntas no etiquetadas que no comparten semántica con las preguntas originales, mediremos cuántas preguntas del corpus que no corresponden a la clase "Otra" son correctamente clasificadas por el sistema.
\end{description}

\subsubsection{Baseline}
Tomaremos como baseline dos métodos:
\begin{description}
    \item[]
\end{description}

\subsubsection{Experimento 1}
\textbf{Hipótesis} El aprendizaje activo obtiene mejores resultados que un clasificador normal utilizando la misma cantidad de datos.
Realizamos este experimento tanto sobre instancias como sobre características. Para eso, debimos elaborar un corpus etiquetado de instacias relacionándolas a una clase dada.

\subsubsection{Experimento 2}
\textbf{Hipótesis} El aprendizaje activo sobre instancias y características obtiene mejores resultados que el aprendizaje activo sobre instancias o características por separado.
% Cómo lo medimos? Por tiempo? Por cantidad de datos etiquedos?
% Por cantidad de acciones que realiza un usuario?
% Probablemente termine siendo el mismo experimento que el anterior

\subsubsection{Experimento 2}
\textbf{Hipótesis} El aprendizaje activo sobre instancias y características obtiene mejores resultados que el aprendizaje activo sobre instancias o características por separado.



Cosas que quedan para hacer:
	-- Agregar los parámetros por consola para emular un usuario
	-- Agregar un registro del progreso del usuario para poder hacer gráficos
