\chapter{Introducción}

% Cuál es el problema
% Por qué es importante
% Qué vamos a hacer al respecto
% Por qué lo abordamos de la forma en que lo hacemos
Los sistemas de respuesta a preguntas son un área naciente del procesamiento del lenguaje natural y particularmente del área de recuperación de información.

\citet{gupta_survey} destacan que existen dos formas principales de buscar la respuesta a una pregunta de un usuario. La primera de ellas consiste de encontrar similitudes semánticas o sintácticas entre la pregunta y documentos de texto que pueden contener evidencias para la respuesta. La segunda, que abordaremos durante este trabajo, traduce la pregunta a un lenguaje formal para luego realizar consultas a una base de datos.

La reciente posibilidad de manejo de grandes volúmenes de datos ha permitido la formación de grades bases de conocimiento públicas y disponibles online. Estas web semánticas u ontologías cambian ampliamente el paradigma utilizado hasta el momento, ya que estructuran los datos y permiten entonces extraer relaciones complejas entre sus entidades, como plantea \citet{ou_entailement}.

Sin embargo, el primer paso para la resolución de una pregunta es la formalización de la misma a partir del texto ingresado por el usuario, independientemente del método de extracción de información empleado a continuación. Aún así, la mayoría de los sistemas se centran en la búsqueda de la respuesta más que en la correcta interpretación de la pregunta, y en general se limitan a textos cortos y preguntas puntuales.
%% http://delivery.acm.org/10.1145/1080000/1073092/p41-ravichandran.pdf?ip=201.212.190.199&id=1073092&acc=OPEN&key=4D4702B0C3E38B35.4D4702B0C3E38B35.4D4702B0C3E38B35.6D218144511F3437&CFID=407135832&CFTOKEN=92333839&__acm__=1408457987_7e50476334404266ebcb5c987e916bf4
%%Learning Surface Text Patterns for a Question Answering System 2002
%% Problemas que en general enfrentan los sistemas de respuesta a preguntas

Una aproximación simple a este problema es la de Quepy, un framework de traducción automática de preguntas en lenguaje natural a un lenguaje de consultas formalizado. El programador define una serie de plantillas para cada tipo de pregunta que el sistema pueda procesar y su correspondiente interpretación en la base de conocimiento elegida.
%Los lenguajes soportados hasta el momento son MQL y SPARQL; ambos permiten consultas posteriores a FreeBase y DBPedia.

Aunque Quepy está diseñado para simplificar la tarea de construcción de dichas reglas, el trabajo necesario para lograr cobertura amplia es todavía prohibitivo por varios motivos:
\begin{itemize}
    \item Las plantillas deben ser desarrolladas por un experto de forma individual.
    \item El poder expresivo de las preguntas que soporte el sistema es lineal con respecto a la cantidad de plantillas generadas.
    \item Existe redundancia de información. Por ejemplo, para las preguntas ``Who are the presidents of Argentina?'' y ``Who are the children of the presidents of Argentina?'' se necesitan dos plantillas que contienen la misma información para resolver ``presidents of X''.
    \item Existen numerosas preguntas que son equivalentes y que no necesariamente se representan con la misma plantilla. Por ejemplo las preguntas ``Where is Angelina Jolie from?'' y ``Where was Angelina Jolie born?'' tienen escencialmente la misma semántica.
    \item Debido a las grandes variaciones del lenguaje natural, se requiere un anotador experto para lograr una cobertura completa de todas las reformulaciones para una misma semántica.
\end{itemize}

De todas las dificultades anteriores nos enfocaremos en las dos últimas ya que la consideramos prioritara y, al solucionarla, podemos ampliar la cobertura de los sistemas construidos sobre Quepy significativamente. Nuestra propuesta es aplicar un clasificador automático sobre las preguntas donde cada clase es una interpretación de Quepy. De esta forma, podemos ligar muchas más reformulaciones de la misma pregunta a su correspondiente semántica y lograr mayor versatilidad.

La originalidad de nuestra aplicación se base en utilizar como características las concordancias parciales con las plantillas de Quepy predefinidas por un programador. Consideramos que identifican claramente los aspectos relevantes que indican la correcta interpretación de la pregunta, y como tal son mejores representaciones.

Para evaluar nuestro sistema consideramos que comenzar con pocos patrones predefinidos nos ayudaría a percibir con más exactitud qué mejora podía generar el clasificador en Quepy. Por ello, y debido a que no existen grandes corpus etiquetados para reformulaciones de preguntas, planteamos que un enfoque de aprendizaje activo es lo más adecuado. El aprendizaje activo, como describe \citet{settles_active_learning_survey}, permite entrenar el aprendedor con menor cantidad de instancias y es beneficioso cuando se cuenta con muchos ejemplos no etiquetadas pero donde la mayoría no son relevantes. En nuestro entorno en particular se da este fenómeno, debido a que en un corpus no anotado estándar pocas de las preguntas caerán dentro de alguna de las clases semánticas de los patrones iniciales.

Un enfoque novedoso que combina todos los conceptos anteriores es el de \citet{dualist} en Dualist. Esta herramienta optimiza el aprendizaje activo no solo preguntado al usuario sobre instancias sino también sobre características de las mismas que las asocian a una clase. Junto con este desarrollo también incluye una serie de investigaciones sobre el rendimiento de tareas de clasificación con usuarios reales y simulados. Es por ello que tomamos como base este trabajo y lo adaptamos con una nueva implementación a nuestro problema.
